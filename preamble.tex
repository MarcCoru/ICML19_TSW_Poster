

\usepackage{microtype}
\usepackage{graphicx}
%\usepackage{subfigure}
\usepackage{booktabs} 
\usepackage{hyperref}
\newcommand{\theHalgorithm}{\arabic{algorithm}}
%\usepackage[accepted]{icml2019}

% may help for figure placements
\usepackage{dblfloatfix}

%\setlength{\abovecaptionskip}{5pt plus 3pt minus 2pt}
%\setcitestyle{numbers,round,citesep={;},aysep={,},yysep={;}}

\usepackage{times}
\usepackage{epsfig}
\usepackage{graphicx}
\usepackage{amsmath}
\usepackage{amssymb}
\usepackage[utf8]{inputenc}
\usepackage{booktabs}
\setlength{\tabcolsep}{5pt}
%\usepackage{subcaption}
\usepackage{tumabbrev}

\usepackage{appendix}

% Include other packages here, before hyperref.

% If you comment hyperref and then uncomment it, you should delete
% egpaper.aux before re-running latex.  (Or just hit 'q' on the first latex
% run, let it finish, and you should be clear).
%\usepackage[breaklinks=true,bookmarks=false]{hyperref}


\usepackage[capitalize]{cleveref}
%\usepackage[square,sort,comma,numbers]{natbib}
%\usepackage{natbib}

%% this hack seems to be nececessary due to incompatibilities of cvpr template and tikz... -> https://tex.stackexchange.com/questions/398223/tikz-gives-error-command-everyshipouthook-already-defined
%\makeatletter
%\@namedef{ver@everyshi.sty}{}
%\makeatother
%% hackend

\usepackage{tikz}
\usepackage{pgfplots}
\usetikzlibrary{positioning, calc,arrows,arrows.meta, fit}
%\usetikzlibrary{arrows.meta,calc,decorations.markings,math,arrows.meta}
\usepgfplotslibrary{groupplots}
\usepgfplotslibrary{fillbetween}
\usepgfplotslibrary{statistics} % provides boxplots
\usepackage{xfrac}
\usetikzlibrary{backgrounds}

%\pgfdeclarelayer{background}
%\pgfdeclarelayer{foreground}
%\pgfsetlayers{background,foreground}

\usetikzlibrary{shapes,snakes}

\newcommand{\tp}{tp}
\newcommand{\tn}{tn}
\newcommand{\fp}{fp}
\newcommand{\fn}{fn}


\usepackage{tumcolors}
\usepackage{tummath}
\newcommand{\yhat}{\hat{\V{y}}}
\newcommand{\ycorrect}{\hat{y}^+}
\newcommand{\thetadelta}{\V{\Theta}_\delta}
\newcommand{\biasdelta}{b_\delta}
\newcommand{\biasclass}{\V{b}_\text{c}}
\newcommand{\thetaclass}{\V{\Theta}_\text{c}}
\newcommand{\thetafeat}{\V{\Theta}_\text{feat}}
\newcommand{\fclass}{f_\text{c}}
\newcommand{\fdelta}{f_\delta}
\newcommand{\ffeat}{f_\text{feat}}
\newcommand{\f}{f}

\newcommand{\rvtime}{T_c} 
\newcommand{\xuptot}{\M{X}_{\rightarrow t}} 
\newcommand{\deltauptot}{\delta_{\rightarrow t}} 
\newcommand{\tstop}{\ensuremath{t_\text{stop}}}
\newcommand{\meantstop}{\ensuremath{\bar{t}_\text{stop}}}
\usepackage[super]{nth}
\usepackage{mathtools}

\definecolor{evalcolor}{HTML}{3F3F3F}
\definecolor{traincolor}{HTML}{B98951}
\definecolor{validcolor}{HTML}{3F4BBE}

\colorlet{colortrain}{tumblue}
\colorlet{colorinfer}{tumblack}

\colorlet{earlinesscolor}{tumblue}
\colorlet{accuracycolor}{tumorange}

\colorlet{stdcolor}{tumbluelight}
\colorlet{mediancolor}{tumorange}
\colorlet{meancolor}{tumblue}

\colorlet{frh01color}{tumgray}
\colorlet{frh02color}{tumorange}
\colorlet{frh03color}{tumblue}
\colorlet{frh04color}{tumblack}

\colorlet{b1color}{tumdiagramaubergine}
\colorlet{b2color}{tumdiagramnavyblue}
\colorlet{b3color}{tumdiagramturquoise}
\colorlet{b4color}{tumdiagramgreen}
\colorlet{b5color}{tumdiagramlimegreen}
\colorlet{b6color}{tumdiagramyellow}
\colorlet{b7color}{tumdiagramsand}
\colorlet{b8color}{tumdiagramredorange}
\colorlet{b8Acolor}{tumdiagramred}
\colorlet{b9color}{tumblack}
\colorlet{b10color}{tumblue}
\colorlet{b11color}{tumdiagramdarkred}
\colorlet{b12color}{tumorange}

% atmospheric bands
%\colorlet{b1color}{tumblack}%tumdiagramaubergine
%\colorlet{b9color}{tumblack}%tumblack
%\colorlet{b10color}{tumblack}%tumblue
%
%%visisble bands
%\colorlet{b2color}{tumblue}%tumdiagramnavyblue
%\colorlet{b3color}{tumblue}%tumdiagramturquoise
%\colorlet{b4color}{tumblue}%tumdiagramgreen
%
%% near infrared bands
%\colorlet{b5color}{tumdiagramred}%tumdiagramlimegreen
%\colorlet{b6color}{tumdiagramred}%tumdiagramyellow
%\colorlet{b7color}{tumdiagramred}%tumdiagramsand
%\colorlet{b8color}{tumdiagramred}%tumdiagramredorange
%\colorlet{b8Acolor}{tumdiagramred}%tumdiagramred

% SWIR bands
%\colorlet{b11color}{tumorange}%tumdiagramdarkred
%\colorlet{b12color}{tumorange}%tumorange

\colorlet{epsilon0color}{tumorange}
\colorlet{epsilon1color}{tumblue}
\colorlet{epsilon10color}{tumblack}

\colorlet{meadowcolor}{tumbluemedium}
\colorlet{wbarleycolor}{tumbluedark}
\colorlet{corncolor}{tumorange}
\colorlet{wheatcolor}{tumgreen}
\colorlet{sbarleycolor}{tumdiagramred}
\colorlet{clovercolor}{tumdiagramturquoise}
\colorlet{triticalecolor}{tumdiagramsand}

\tikzstyle{rnn}=[draw,circle, inner sep=.1em]
\tikzstyle{norm}=[rounded corners,draw]
\tikzstyle{annot}=[rounded corners, fill=tumblue!20]
\tikzstyle{infer}=[-stealth, shorten >=.0em, shorten <=.0em, colorinfer]
\tikzstyle{loss}=[fill=tumblue!10, rounded corners, font=\small]
\tikzstyle{grad}=[colortrain]

\newcommand{\ptoffset}{\varepsilon}

\tikzstyle{test} = [thick]
\tikzstyle{train} = [thin, dotted]

\usepackage[inline]{enumitem}
\setenumerate{label=(\roman*),itemsep=3pt,topsep=3pt}

%\setlength{\belowcaptionskip}{-10pt}
%\usepackage{titlesec}
%\titlespacing{\section}{0pt}{10pt}{3pt}

\usetikzlibrary{external,pgfplots.dateplot}
\tikzexternalize[prefix=tikz/]
%\tikzexternalize
\tikzexternaldisable

\usepackage[eulergreek]{sansmath}
\pgfplotsset{
	y tick label style={/pgf/number format/.cd,%
		scaled y ticks = false,
		set thousands separator={},
		fixed},
	x tick label style={/pgf/number format/.cd,%
		scaled x ticks = false,
		set decimal separator={,},
		fixed},
	tick label style = {font=\tiny\sansmath\sffamily},
	every axis label = {
		font=\tiny\sansmath\sffamily},
	every axis/.append style={
		axis lines=left, 
		enlargelimits, 
		thick},
	legend style = {font=\tiny\sansmath\sffamily, draw=none, rounded corners, fill opacity=.5, text opacity=1},
	label style = {font=\tiny\sansmath\sffamily},
	grid style={line width=.1pt, draw=gray!10},
	major grid style={line width=.2pt,draw=tumgraylight},
}

%\let\tempone\itemize
%\let\temptwo\enditemize
%\renewenvironment{itemize}{\tempone\addtolength{\itemsep}{-.5\baselineskip}}{\temptwo}

\tikzstyle{circ} = [circle, draw=white, fill=tumblue, inner sep=.08em]
\newcommand{\fcn}{
	\begin{tikzpicture}[scale=.5, rotate=0, baseline=-.25em, inner sep=1pt]
	\node[circ](a0) at (0,-1){};
	\node[circ](a1) at (0,0){};
	\node[circ](a2) at (0,1){};
	
	\node[circ](b0) at (1,-0.5){};
	\node[circ](b1) at (1,0.5){};
	
	\draw[-] (a0) -- (b0);
	\draw[-] (a1) -- (b0);
	\draw[-] (a2) -- (b0);
	
	\draw[-] (a0) -- (b1);
	\draw[-] (a1) -- (b1);
	\draw[-] (a2) -- (b1);
	
	\end{tikzpicture}
}

\newcommand{\hidden}[1]{
	\begin{tikzpicture}[scale=.1, baseline=-.25em]	
	%\draw[step=1.0,black,thin] (0,0) grid (#1,1);
	\foreach \i in {1,...,#1}{
		\node[circle, draw=white, fill=tumbluelight, inner sep=1pt] at (\i,0){};
	}
	\end{tikzpicture}
}

\newcommand{\drawvector}[1]{
	\begin{tikzpicture}[scale=.1, baseline=-.25em]	
	%\draw[step=1.0,black,thin] (0,0) grid (#1,1);
	\foreach \i in {1,...,#1}{
		\node[circ] at (\i,0){};
	}
	\end{tikzpicture}
}


\tikzstyle{druschdatum} = [thin, star,star points=3, star point ratio=0.5, inner sep=.15em, draw=tumwhite, fill=tumblue]

\newcommand{\druschdatum}{
\begin{tikzpicture}[scale=2, baseline=-.25em, inner sep=0]
\node[druschdatum, inner sep=.25em]{};
\end{tikzpicture}
}

\tikzstyle{box} = [rounded corners=.5em, inner sep=1em]


\input{images/confmat.tikz}
	\newcommand{\experiment}[1]{
	
	%	
	\begin{subtable}[b]{.33\textwidth}
		\tiny	
		\input{#1/npy/table.tex}
		\caption{per-class accuracy metrics}
		\label{tab:perclass:tab}
	\end{subtable}
	\begin{subfigure}[b]{.33\textwidth}
		\confmat{#1/npy/confmat_flat.csv}{3}{1}
		\caption{precision}
		\label{fig:perclass:prec}
	\end{subfigure}
	\begin{subfigure}[b]{.33\textwidth}
		\confmat{#1/npy/confmat_flat.csv}{4}{1}
		\caption{recall}
		\label{fig:perclass:rec}
	\end{subfigure}
	
	%	\begin{subfigure}[b]{.33\textwidth}
	%		\confmat{#1/npy/confmat_flat.csv}{2}{5}
	%		\caption{Häufigkeit in Potenzen von 10}
	%	\end{subfigure}
}

\newcommand{\wave}{
	\begin{tikzpicture}[xscale=.01,yscale=.1]
	\draw[-,fill=white] plot[domain=0:10*pi,smooth] (\x,{sin(\x r)});
	\end{tikzpicture}
}
